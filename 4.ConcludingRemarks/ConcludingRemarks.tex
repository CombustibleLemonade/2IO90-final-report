\documentclass[crop=false,a4paper,oneside,11pt]{standalone}
\usepackage{a4wide,graphicx,fancyhdr,amsmath,amssymb,float,graphicx,color,geometry,xcolor,titlesec,colortbl,tabu}
\usepackage[parfill]{parskip}
\usepackage[nodayofweek]{datetime}
%----------------------- Macros and Definitions --------------------------

%fast change of things
\newcommand{\mysubject}{2IO70 DBL Embedded Systems}
\newcommand{\myassignment}{Group 12}

%\definecolor{titlepagecolor}{cmyk}{1,.60,0,.40}
%\definecolor{namecolor}{cmyk}{1,.50,0,.10}


\setlength\headheight{20pt}
\addtolength\topmargin{-10pt}
\addtolength\footskip{20pt}

% Define light and dark Microsoft blue colours
\definecolor{MSBlue}{rgb}{.204,.353,.541}
\definecolor{MSLightBlue}{rgb}{.31,.506,.741}
\arrayrulecolor{MSLightBlue}

% Set formats for each heading level

\titleformat*{\section}{\Large\bfseries\sffamily\color{MSBlue}}
\titleformat*{\subsection}{\large\bfseries\sffamily\color{MSLightBlue}}

%date format
\newdateformat{mydate}{\monthname[\THEMONTH] \THEYEAR}

\fancypagestyle{plain}{%
\fancyhf{}
\renewcommand{\headrulewidth}{0pt}
\renewcommand{\footrulewidth}{0pt}
}

\pagestyle{fancy}
\fancyhf{}
\fancyfoot[CO] {\thepage}
\renewcommand{\headrulewidth}{0pt}
\renewcommand{\footrulewidth}{0pt}


%--------------------------------- Text ----------------------------------
\setcounter{secnumdepth}{0}
\begin{document}

\section{Concluding Remarks}
We created algorithms for 3 placement models using a quad tree and rules proposed by Alexander Wolff in \emph{Automated Label Placement in Theory and Practice} [1]. The 2-position algorithm and the 4-position algorithm are similar and run in $O(n^2)$. In practice we have found that the running time of the these algorithms are $\Omega(n\log n)$ and $O(n^2)$. The solutions that these algorithms produce are not always optimal. The 1-slider algorithm runs in $O(n^3)$ time, this is also what we found in practice and also does not produce optimal solutions. Our algorithms work well when there are few candidates that have overlaps. The 4-position algorithm places the highest percentage of labels and the 2-position algorithm places the lowest percentage of labels. All three algorithms do not perform well when there are many candidates with overlaps.

\subsection{Future work}
Our goal was to make an algorithm that approaches an optimal solution. Some future research could be done in finding algorithms that are even closer to an optimal solution. Another improvement that could be made is a better implementation of rule L2 for the 2-position and 4-position algorithms. We currently use three rules (L1, L2 and L3). In future work, it could be investigated whether there are more rules that can be applied without spoiling the optimal solution. Some other improvements could be made in all of the algorithms when many collisions occur as the algorithms currently have a high running time when this occurs. A better balance between storing collisions and checking them over and over would probably greatly improve the running time (2-position and 4-position store collisions, overusing the memory and slowing everything down. 1-slider checks collisions again every time, which also increases the running time). 

\end{document}
